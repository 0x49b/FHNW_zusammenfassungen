\documentclass{beamer}
\usetheme{FHNW}
\usepackage[utf8x]{inputenc}
\usepackage[german]{babel}
\usepackage{color}
\usepackage{xcolor}
\usepackage{listings}
\usepackage{caption}
\DeclareCaptionFont{white}{\color{white}}
\DeclareCaptionFormat{listing}{\colorbox{gray}{\parbox{\textwidth}{#1#2#3}}}
\captionsetup[lstlisting]{format=listing,labelfont=white,textfont=white}
\lstdefinestyle{JavaStyle}{
 language=java,
 basicstyle=\footnotesize\ttfamily, % Standardschrift
 numbers=left, % Ort der Zeilennummern
 numberstyle=\tiny, % Stil der Zeilennummern
 stepnumber=5, % Abstand zwischen den Zeilennummern
 numbersep=5pt, % Abstand der Nummern zum Text
 tabsize=2, % Groesse von Tabs
 extendedchars=true, %
 breaklines=true, % Zeilen werden Umgebrochen
 frame=b,
 %commentstyle=\itshape\color{LightLime}, Was isch das? O_o
 %keywordstyle=\bfseries\color{DarkPurple}, und das O_o
 basicstyle=\footnotesize\ttfamily,
 stringstyle=\color[RGB]{42,0,255}\ttfamily, % Farbe der String
 keywordstyle=\color[RGB]{127,0,85}\ttfamily, % Farbe der Keywords
 commentstyle=\color[RGB]{63,127,95}\ttfamily, % Farbe des Kommentars
 showspaces=false, % Leerzeichen anzeigen ?
 showtabs=false, % Tabs anzeigen ?
 xleftmargin=17pt,
 framexleftmargin=17pt,
 framexrightmargin=5pt,
 framexbottommargin=4pt,
 showstringspaces=false % Leerzeichen in Strings anzeigen ?
}
\newcommand\Fontvi{\fontsize{6}{7.2}\selectfont}



%\lstdefinestyle{htmlStyle} {
%   language=html,
%   basicstyle=\scriptsize\ttfamily,
%   keywordstyle=\bfseries\ttfamily,
%   commentstyle=\color{gray}\ttfamily,
%   escapechar=| % Escape to LaTeX between |...|
%}
\begin{document}
\title{FHNW APSI Lab 2 : Html Form}   
\author{Jan Fässler, Fabio Oesch} 

\date{\today} 

\frame{\titlepage} 

\begin{frame}
\section*{Übersicht}
\tableofcontents
\end{frame} 

\section{Robuste Programmierung im Controller und Model} 
\begin{frame}
 \subsection{SQL-Injection \& XSS Attacken}
 \textbf{SQL-Injection \& XSS Attacken}
\end{frame}

\subsection{Validierung}
\begin{frame}
 \textbf{Validierung}\\
 \begin{itemize}
  \item Regexausdruck der die Eingaben überprüft
  \item E-Mail Adresse
  \item Postleitzahl wird mit Post.ch
  \item Passwort
  \item Model wird geschützt, dass 2 verschiedene Konstruktoren existieren
 \end{itemize}
\end{frame}

\section{SSL}
\begin{frame}
 \textbf{SSL}\\
\end{frame}



\begin{frame}
\section{Demo}
 \huge Demo
\end{frame}

\begin{frame}
\section*{Ende}
 \huge Danke für Ihre Aufmerksamkeit.
\end{frame}

\end{document}

